Es gilt:

\begin{minipage}{0.45\textwidth}

\mainformular{ $F_G = \frac{F_o}{1 + F_o}$} \\
\mainformular{ $F_G = \frac{Z_o}{Z_o + N_o}$} \\
\mainformular{ $F_o = F_R \cdot F_S$} \\

\end{minipage}
\begin{minipage}{0.45\textwidth}

\legende{
$F_G$ & geschlossener Kreis  & & \\
$Z_o$ & Zähler offener Kreis  & & \\
$N_o$ & Nenner offener Kreis  & & \\
$F_o$ & offener Kreis  & & \\
$F_G$ & geschlossener Kreis  & & \\

}
\end{minipage}

\subsubsection{Hurwitz-Kriterium}
charakteristische Gleichung des geschl. Regelkreises:
$$ a_mp^m + a_{m-1}p^{m-1}+ ... + a_1p+ a_0 = 0$$

notwendige Bedingung: alle Koeffizienten der charakteristischen Gleichung des geschlossenen Regelkreises müssen vorhanden und positives Vorzeichen haben. \\

hinreichende Bedingung: Alle Hauptabschnitssdeterminanten $D_i$
der Hurwitzdeterminantte H müssen positiven Wert haben.\\

\begin{minipage}{0.45\textwidth}

\mainformular{ $D_2 = a_1\cdot a_2 - a_3 \cdot a_0$} \\
\end{minipage}
\begin{minipage}{0.45\textwidth}

\legende{
$D_2$ & Determinante rel. für System 3.Ord.  & & \\

}
\end{minipage}

\subsubsection{Niquist-Kriterium}
Der geschlossene Regelkreis ist stabil, wenn der kritische Punkt (-1,0) links der Ortskurve $F_o(j\omega)$ seines offenen Kreises liegt.

\begin{minipage}{0.45\textwidth}

\mainformular{ $F_o(j\omega) = \frac{K}{A(j\omega)+ j B(j\omega)}$} \\
$\omega_k \Rightarrow B(\omega) = 0$ \\
$\frac{K}{A(\omega_k)} > -1 $
\end{minipage}
\begin{minipage}{0.45\textwidth}

\legende{
$F_o(j\omega)$ & Übertragungsfkt. offenen Kreis  & & \\
& Berechnungen zum Prüfen d. Stabilität & & \\
}
\end{minipage}
