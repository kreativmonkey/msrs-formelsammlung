Es gilt:

\begin{minipage}{0.45\textwidth}

\mainformular{ $F_G = \frac{F_o}{1 + F_o}$} \\
\mainformular{ $F_G = \frac{Z_o}{Z_o + N_o}$} \\
\mainformular{ $F_o = F_R \cdot F_S$} \\

\end{minipage}
\begin{minipage}{0.45\textwidth}

\legende{
$F_G$ & geschlossener Kreis  & & \\
$Z_o$ & Zähler offener Kreis  & & \\
$N_o$ & Nenner offener Kreis  & & \\
$F_o$ & offener Kreis  & & \\
$F_G$ & geschlossener Kreis  & & \\

}
\end{minipage}

\subsubsection{Hurwitz-Kriterium}
charakteristische Gleichung des geschl. Regelkreises:
$$ a_mp^m + a_{m-1}p^{m-1}+ ... + a_1p+ a_0 = 0$$

notwendige Bedingung: alle Koeffizienten der charakteristischen Gleichung des geschlossenen Regelkreises müssen vorhanden und positives Vorzeichen haben. \\

hinreichende Bedingung: Alle Hauptabschnitssdeterminanten $D_i$
der Hurwitzdeterminantte H müssen positiven Wert haben.\\

\begin{minipage}{0.45\textwidth}

\mainformular{ $D_2 = a_1\cdot a_2 - a_3 \cdot a_0$} \\
\end{minipage}
\begin{minipage}{0.45\textwidth}

\legende{
$D_2$ & Determinante rel. für System 3.Ord.  & & \\

}
\end{minipage}

\subsubsection{Allgemeine Lösung Hurwitz-Kriterium für Systeme 3. Ortnung}

\begin{minipage}{.45\textwidth}
 \begin{equation} \label{Fstrecke}
  F_S(p) = \cfrac{V_S}{(T_1p+1)T_0p} 
\end{equation}
\end{minipage}
\begin{minipage}{.45\textwidth}
\begin{equation} \label{Fregler}
  F_R(p) = \cfrac{V_R}{T_Rp+1} 
\end{equation}
\end{minipage}

\begin{equation} \label{eq1}
\begin{split}
 F_0(p) &= F_S(p) \cdot F_R(p) \\
 &=  \cfrac{V_S}{(T_1p+1)T_0p} \cdot \cfrac{V_R}{T_Rp+1} \\
 &= \cfrac{V_S\cdot V_R}{((T_1p+1)T_0p) \cdot (T_Rp+1)} \\
 &= \cfrac{V_S\cdot V_R}{(T_0T_1p^2+T_0p) \cdot (T_Rp+1)} \\
 &= \cfrac{V_S\cdot V_R}{T_0T_1T_Rp^3+T_0T_Rp^2+T_0T_1p^2+T_0p} \\
\end{split}
\end{equation}
\begin{equation} \label{eq1}
\begin{split}
 G_W(p) &=  \cfrac{F_0(p)}{1+F_0(p)} \\
 &= \cfrac{\cfrac{V_S\cdot V_R}{T_0T_1T_Rp^3+T_0T_Rp^2+T_0T_1p^2+T_0p}}{1+\cfrac{V_S\cdot V_R}{T_0T_1T_Rp^3+T_0T_Rp^2+T_0T_1p^2+T_0p}} \\
 &= \cfrac{\cfrac{V_S\cdot V_R}{T_0T_1T_Rp^3+T_0T_Rp^2+T_0T_1p^2+T_0p}}{1+\cfrac{T_0T_1T_Rp^3+T_0T_Rp^2+T_0T_1p^2+T_0p+V_S\cdot V_R}{T_0T_1T_Rp^3+T_0T_Rp^2+T_0T_1p^2+T_0p}} \\
 &= \cfrac{V_S\cdot V_R}{p^3(T_0T_1T_R)+p^2(T_0T_R+T_0T_1)+p(T_0)+V_SV_R}
\end{split}
\end{equation}

Eine notwendige Bedingung die erfüllt sein muss lautet: Alle $a_i$ sind vorhanden und $a_i > 0$. Dies ist im gezeigten Beispiel gegeben.


\begin{equation} \label{eg3}
\begin{split}
 G_W(p) &=  \cfrac{F_0(p)}{1+F_0(p)} \\
 &= \cfrac{\cfrac{V_S\cdot V_R}{T_0T_1T_Rp^3+T_0T_Rp^2+T_0T_1p^2+T_0p}}{1+\cfrac{V_S\cdot V_R}{T_0T_1T_Rp^3+T_0T_Rp^2+T_0T_1p^2+T_0p}} \\
 &= \cfrac{\cfrac{V_S\cdot V_R}{T_0T_1T_Rp^3+T_0T_Rp^2+T_0T_1p^2+T_0p}}{1+\cfrac{T_0T_1T_Rp^3+T_0T_Rp^2+T_0T_1p^2+T_0p+V_S\cdot V_R}{T_0T_1T_Rp^3+T_0T_Rp^2+T_0T_1p^2+T_0p}} \\
 &= \cfrac{V_S\cdot V_R}{p^3(T_0T_1T_R)+p^2(T_0T_R+T_0T_1)+p(T_0)+V_SV_R}
\end{split}
\end{equation}

Bei Systemen 3. Ortnung, welche die notwendigen Bedingungen erfüllen folgt die Berechnungen von $D_2$ nach:

\begin{equation} \label{eg3}
 \begin{split}
 D_2 &= a_1a_2-a_0a_3 \\
 &= T_0 \cdot T_0(T_1+T_R) - V_SV_R \cdot T_0T_1T_R \\
 &= T_0^2 (T_1+T_R) - V_SV_RT_0T_1T_R 
\end{split}
\end{equation}

Die Bedingung zur Erfüllung des Hurwitz-Kriteriums lautet $T_0^2 (T_1+T_R) - V_SV_RT_0T_1T_R  > 0$. Auflösung nach $T_R$:

\begin{equation} \label{eg3}
 \begin{split}
 T_0^2T_1+T_0^2T_R - V_SV_RT_0T_1T_R  &> 0 \\
 T_0^2T_1   &> V_SV_RT_0T_1T_R - T_0^2T_R\\
 T_0^2T_1  &> T_R(V_SV_RT_0T_1 - T_0^2) \\
 \cfrac{T_0^2T_1}{V_SV_RT_0T_1 - T_0^2} &> T_R \\
 \cfrac{T_0T_1}{V_SV_RT_1 - T_0} &> T_R
\end{split}
\end{equation}

\subsubsection{Niquist-Kriterium}
Der geschlossene Regelkreis ist stabil, wenn der kritische Punkt (-1,0) links der Ortskurve $F_o(j\omega)$ seines offenen Kreises liegt.

\begin{minipage}{0.45\textwidth}

\mainformular{ $F_o(j\omega) = \frac{K}{A(j\omega)+ j B(j\omega)}$} \\
$\omega_k \Rightarrow B(\omega) = 0$ \\
$\frac{K}{A(\omega_k)} > -1 $
\end{minipage}
\begin{minipage}{0.45\textwidth}

\legende{
$F_o(j\omega)$ & Übertragungsfkt. offenen Kreis  & & \\
& Berechnungen zum Prüfen d. Stabilität & & \\
}
\end{minipage}

\subsubsection{Allgemeines Niquist-Kriterium}

Formale Voraussetzung:

\begin{itemize}
 \item Totzeit $T_z \ge 0$ (Wenn nicht vorhanden ist dieses Kriterium gegeben.)
 \item $F_0(s)$: Zählergrad $<$ Nennergrad
\end{itemize}

Der geschlossene Regelkreis ist stabil, wenn die Winkeländerung der Ortskurve von -1 aus betrachtet folgender Winkeländerung entspricht:

\begin{minipage}{0.45\textwidth}

\mainformular{ $W = \pi \left( r_0 + \cfrac{a_0}{2} \right)$} \\
\end{minipage}
\begin{minipage}{0.45\textwidth}

\legende{
$W$ & Winkeländerung  & & \\
$r_0$ & Anzahl reeler Pole mit pos. Realteil & & \\
$a_0$ & Anzahl Pole auf der j-Achse & & \\
}
\end{minipage}

\subsubsection{Allgemeine Lösung 3. Ortnung}

\begin{minipage}{.45\textwidth}
 \begin{equation} \label{Fstrecke}
  F_S(p) = \cfrac{V_S}{(T_1p+1)T_0p} 
\end{equation}
\end{minipage}
\begin{minipage}{.45\textwidth}
\begin{equation} \label{Fregler}
  F_R(p) = \cfrac{V_R}{T_Rp+1} 
\end{equation}
\end{minipage}

\begin{equation} \label{eg1}
\begin{split}
 F_0(p) &= F_S(p) \cdot F_R(p) \\
 &=  \cfrac{V_S}{(T_1p+1)T_0p} \cdot \cfrac{V_R}{T_Rp+1} \\
 &= \cfrac{V_S\cdot V_R}{((T_1p+1)T_0p) \cdot (T_Rp+1)} \\
 &= \cfrac{V_S\cdot V_R}{(T_0T_1p^2+T_0p) \cdot (T_Rp+1)} \\
 &= \cfrac{V_S\cdot V_R}{T_0T_1T_Rp^3+T_0T_Rp^2+T_0T_1p^2+T_0p} \\
 &= \cfrac{V_S\cdot V_R}{p^2(T_0T_1T_Rp+T_0T_R+T_0T_1+\frac{T_0}{p})} \\
 \end{split}
\end{equation}

\begin{equation} \label{eg2}
\begin{split}
 F_0(jw) &= F_S(jw) \cdot F_R(jw) \\
 &= \cfrac{V_S V_R}{j^2w^2(T_0T_1T_Rjw+T_0T_R+T_0T_1+\frac{T_0}{jw})} \\
 &= \cfrac{V_S V_R}{-T_0T_1T_Rjw^3-T_0T_Rw^2-T_0T_1w^2+T_0jw} \\
 &= \cfrac{V_S V_R}{w^2(-T_0T_R-T_0T_1)+j(T_0w-T_0T_1T_Rw^3)} \\
 \end{split}
\end{equation}

\begin{equation} \label{eq3}
  B(w) = 0 = T_0w-T_0T_1T_Rw^3 = w(T_0-T_0T_1T_Rw^2) 
\end{equation}

\begin{minipage}{.45\textwidth}
\begin{equation}
 w_0=0
\end{equation}
\end{minipage}
\begin{minipage}{.45\textwidth}
\begin{equation} \label{eq4}
\begin{split}
  0 &= T_0-T_0T_1T_Rw^2 \\
  T_0T_1T_Rw^2 &= T_0 \\
  w^2 &= \cfrac{T_0}{T_0T_1T_R} = \cfrac{1}{T_1T_R} \\
  w_{1,2} &= \pm \cfrac{1}{\sqrt{T_1T_R}}
\end{split}
\end{equation}
\end{minipage}

Da negative Werte keinen Sinn ergeben, wird $w_2$ ignoriert. Nun wird $w_1 = w_k = \cfrac{1}{\sqrt{T_1T_R}}$ in die folgende Formel eingesetzt:

\begin{minipage}{.45\textwidth}
\begin{equation} \label{k}
  K = V_RV_S 
\end{equation}
\end{minipage}
\begin{minipage}{.45\textwidth}
 \begin{equation} \label{aw}
  A(w) = w^2(-T_0T_R-T_0T_1)
\end{equation}
\end{minipage}

\begin{equation} \label{eq5}
\begin{split}
 F_0(jw_k) &= \cfrac{K}{A(w_k)} = \cfrac{V_RV_S}{w_k^2(-T_0T_R-T_0T_1)} \\
 &= \cfrac{V_RV_S}{(\cfrac{1}{\sqrt{T_1T_R}})^2(-T_0T_R-T_0T_1)} \\
 &= \cfrac{V_RV_S}{(\cfrac{1}{T_1T_R})(-T_0T_R-T_0T_1)} \\
 &= \cfrac{V_RV_S}{\cfrac{-T_0T_R}{T_1T_R}-\cfrac{T_0T_1}{T_1T_R}} = \cfrac{V_RV_S}{-\cfrac{T_0T_R-T_0T_1}{T_1T_R}} = \cfrac{V_RV_S}{-\cfrac{T_R-T_1}{T_1T_R}} = -\cfrac{V_RV_ST_1T_R}{T_0T_R+T_0T_1} \\
 \end{split}
\end{equation}

Nach dem Nyquist Kriterium muss nun $-\cfrac{V_RV_ST_1T_R}{T_0T_R+T_0T_1} > -1$ gelten.

\begin{equation} \label{eq5}
\begin{split}
-\cfrac{V_RV_ST_1T_R}{T_0T_R+T_0T_1} &> -1 \\
\cfrac{V_RV_ST_1T_R}{T_0T_R+T_0T_1} &< 1 \\
V_RV_ST_1T_R &< T_0T_R+T_0T_1 \\
V_RV_ST_1T_R-T_0T_R &< T_0T_1 \\
T_R(V_RV_ST_1-T_0) &< T_0T_1 \\
T_R &< \cfrac{T_0T_1}{V_RV_ST_1-T_0} \\
 \end{split}
\end{equation}
